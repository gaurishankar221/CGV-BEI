% ==========================================================
% SAMPLE REPORT (COMPUTER GRAPHICS) — SAME TEMPLATE STRUCTURE
% ==========================================================
% NOTE:
% 1) This is a “sample content” report that fits your template layout.
% 2) It assumes you have these files:
%    - includes/coverpage.tex
%    - includes/copyright.tex
%    - includes/acknowledgemt.tex
%    - includes/abstract.tex
%    - includes/acronyms.tex
%    - includes/ref.bib
%    - images/ (with some sample images)
%
% 3) If you compile with minted, use:
%    pdflatex -shell-escape main.tex
% ==========================================================

\documentclass[a4paper, 12pt]{report}

%---PREAMBLE BEGINS HERE---

\usepackage{graphicx}
\usepackage[a4paper]{geometry}
\usepackage{titlesec}
\usepackage{minted}
\usepackage{amsmath, amsthm, amsfonts}
\usepackage{mathptmx}
\usepackage[skip=18pt, indent=40pt]{parskip}
\usepackage{setspace}
\usepackage[acronym, nomain, section=section]{glossaries}
\usepackage{glossaries-extra}
\usepackage[nottoc]{tocbibind}
\usepackage{ifthen}
\usepackage{tocloft}
\usepackage{enumitem}
\usepackage{fancyhdr}
\usepackage{hyperref}

\fancyhf{}
\fancyfoot[C]{\thepage}
\renewcommand{\headrulewidth}{0pt}
\renewcommand{\footrulewidth}{0pt}

\setlist[itemize]{noitemsep, topsep=0pt}
\setlist[enumerate]{noitemsep, topsep=0pt}

% Creating new style acronyms for tabulating acronyms
\newglossarystyle{acronyms}{%
  \renewenvironment{theglossary}{\begin{tabular}{ll}}{\end{tabular}}%
  \renewcommand*{\glossaryheader}{}%
  \renewcommand*{\glsgroupheading}[1]{}%
  \renewcommand*{\glossentry}[2]{\glsentryitem{##1}\glstarget{##1}{\glossentryname{##1}} & \glossentrydesc{##1}\tabularnewline}%
  \renewcommand*{\subglossentry}[3]{\glstarget{##2}{\glossentryname{##2}} & \glossentrydesc{##2}\tabularnewline}%
  \renewcommand*{\glsgroupskip}{}%
}
\makeglossaries

% defining the page margins
\newgeometry{
    top=1in,
    bottom=1in,
    inner=1.5in,
    outer=1in,
}

% defining variables, DO CHANGE THIS BEFORE COMPILE
\newcommand{\projectTitle}{Computer Graphics Mini Project Report}
\newcommand{\projectDate}{February, 2026}
\newcommand{\projectAuthor}{Seven Shades}

% Paragraph formating for heading and normal paragraphs
\titleformat{\chapter}
   {\fontsize{12pt}{12pt}\bfseries\uppercase}{\thechapter.}{1em}{}[\vspace{-\parskip}]
\titlespacing*{\chapter}{0pt}{0pt}{12pt}

\titleformat{\section}
   {\fontsize{12pt}{12pt}\bfseries}{\thesection}{1em}{}
\titlespacing*{\section}{0pt}{18pt}{10pt}

\titleformat{\subsection}
   {\fontsize{12pt}{12pt}\bfseries}{\thesubsection}{1em}{}
\titlespacing*{\subsection}{0pt}{18pt}{8pt}

\titleformat{\subsubsection}
   {\fontsize{12pt}{12pt}\bfseries}{\thesubsubsection}{1em}{}
\titlespacing*{\subsubsection}{0pt}{0pt}{6pt}
\setcounter{secnumdepth}{3}

% Customizing chapter section in Table of Contents
\renewcommand{\cftfigpresnum}{Figure\ }
\setlength{\cftfignumwidth}{5em}

\renewcommand{\cfttabpresnum}{Table\ }
\setlength{\cfttabnumwidth}{4.5em}

\renewcommand{\contentsname}{Table of Contents}
\setlength{\cftbeforechapskip}{6pt}
\renewcommand{\cftchapaftersnum}{.}
\renewcommand{\cftchapleader}{\cftdotfill{\cftdotsep}}
\renewcommand{\cftchapfont}{\bfseries}

\renewcommand{\cfttoctitlefont}{\fontsize{12pt}{12pt}\bfseries}
\setlength{\cftbeforetoctitleskip}{0pt}
\setlength{\cftaftertoctitleskip}{12pt}

\renewcommand{\cftloftitlefont}{\fontsize{12pt}{12pt}\bfseries}
\setlength{\cftbeforeloftitleskip}{20pt}
\setlength{\cftafterloftitleskip}{20pt}

\renewcommand{\cftlottitlefont}{\fontsize{12pt}{12pt}\bfseries}
\setlength{\cftbeforelottitleskip}{20pt}
\setlength{\cftafterlottitleskip}{20pt}

\hypersetup{
  pdfauthor={\projectAuthor},
  pdftitle={\projectTitle}
}

\title{\projectTitle}
\author{\projectAuthor}
\date{\projectDate}

\doublespacing

%---PREAMBLE ENDS HERE---
\begin{document}

\newacronym{ioe}{I.O.E.}{Institute of Engineering}





%must be in alphabetical order

%---MAIN DOCUMENT BEGINS HERE---

\thispagestyle{empty}   % removing page number

% inserting the logo of Tribhuwan University
\begin{figure}
    \centering
    \includegraphics[scale=0.125]{images/tu_logo.png}
\end{figure}

\begin{center}

\textbf{TRIBHUVAN UNIVERSITY}\\
\textbf{INSTITUTE OF ENGINEERING}\\

\textbf{A Project Report}\\
\textbf{On}\\
 \textbf{\ Interactive Visualization and Learning of Bézier Curves Using Python}
\vspace{18pt}

\textbf{Submitted By:}\\
Sayal Shrestha (HCE081BEI041)\\

\textbf{Submitted To:}\\
Department of Electronics and Computer Engineering  \\
Himalaya College of Engineering \\
Chyasal,Lalitpur

\vspace{18pt}
Feb-14,2025

\end{center}


\onehalfspacing

\pagenumbering{roman}
\pagestyle{plain}

\include{includes/copyright}
\chapter*{acknowledgement}
\addcontentsline{toc}{chapter}{ACKNOWLEDGMENT} % Manually add acknowledgment to TOC

I express my sincere gratitude to all those who have supported and guided me throughout the process of conducting this report on Interactive Visualization and Learning of Bézier Curves Using Python. This endeavor would not have been possible without their valuable contributions and assistance.

\noindent This project helped me strengthen my understanding of mathematical visualization, curve modeling, and interactive graphics using Python. I would also like to thank my instructor, Sushant Pandey for the guidance and conceptual clarity provided throughout the course. His insightful feedback and continued encouragement have been invaluable in refining the scope and focus of this project.

\noindent This work would not have been possible without the collective efforts of these individuals and organizations. Although any shortcomings in this report are solely my responsibility, their contributions have significantly enriched its content.

\noindent Thank you.

\noindent Sayal Shrestha (HCE081BEI041)\\


\newpage

\chapter*{abstract}
\addcontentsline{toc}{chapter}{ABSTRACT} % Manually add glossary to TOC



\noindent This project presents an interactive Computer Graphics application developed using Python, NumPy, Matplotlib, and Tkinter. The system visualizes fundamental mathematical and graphical concepts including complex number transformations, 2D Bézier curves using De Casteljau’s algorithm, and 3D Bézier curves using Bernstein polynomials.

\noindent The project is divided into modular components controlled by a central UI launcher. It integrates real-time interaction, parameter sliders, draggable control points, quiz-based learning, and parametric equation generation.



\newpage


\tableofcontents
\newpage

\listoffigures
\newpage

\listoftables
\newpage

\addcontentsline{toc}{chapter}{List of Abbreviations}
\printglossary[type=acronym,style=acronyms]
\newpage

\pagenumbering{arabic}
\pagestyle{fancy}

% ==========================================================
% CHAPTER 1: INTRODUCTION
% ==========================================================
\chapter{Introduction}

Computer Graphics (CG) focuses on creating, manipulating, and displaying visual content using computers. Modern CG systems rely on mathematical models of geometry, lighting, and imaging to convert a 3D scene into a 2D image. This report presents a mini project titled \textbf{Interactive Visualization and Learning of Bezier Curves Using Python}, which demonstrates core CG concepts such as interactive visualization of complex number
transformations and Bezier curves using Python..

\section{Background Introduction}
Bézier curves are defined using Bernstein polynomials and are widely used in CAD, animation, and UI rendering.
A Bézier curve of degree $n$ is defined as:
\[
P(t) = \sum_{k=0}^{n} \binom{n}{k} t^k (1-t)^{n-k} P_k
\]
Complex numbers provide a natural way to represent 2D rotation and scaling:
\[
z = r e^{i\theta}
\]
\section{Motivation}
Many students understand formulas theoretically but struggle to visualize their geometric meaning. The motivation behind this project is:
\begin{itemize}
    \item To convert mathematical exercise questions into interactive visual demonstrations
    \item To build geometric intuition
    \item To combine learning and interactivity
    \item To create a modular and expandable CG system
\end{itemize}

\section{Objectives}
The main objectives of the project are listed below:
\begin{itemize}
    \item Implement interactive visualization of complex number transformations
    \item Implement 2D Bézier curves using De Casteljau algorithm
    \item Implement 3D Bézier curves using Bernstein basis
    \item Display parametric equations of Bézier curves
    \item Integrate quiz-based conceptual validation
    \item Create a modular UI system to launch different modules
\end{itemize}

\section{Scope}
This project covers fundamental CG concepts used in 2D rendering:
\begin{itemize}
    \item 2D complex plane transformations
    \item Parametric curve generation
    \item Interactive sliders and control point dragging
    \item Real-time graphical updates
    \item 3D curve plotting
    \item Educational MCQ-based reinforcement
\end{itemize}
It does not include a full exam oriented mcqs, procedural derivation, or GPU based rendering. Those are suggested as future enhancements.

% ==========================================================
% CHAPTER 2: LITERATURE REVIEW
% ==========================================================
\chapter{Literature Review}

\section{Complex Numbers}
Multiplying a complex number by $e^{i\theta}$ rotates it by angle $\theta$. Multiplying by $r$ scales its magnitude. This provides an elegant method for 2D geometric transformations without explicitly using matrices.

\section{Bézier curves}
Bézier curves were introduced by Pierre Bézier for automobile body design. They are based on Bernstein polynomials and provide smooth parametric curves controlled by a set of control points.
Two major approaches exist for computing Bézier curves:
    \item Bernstein Polynomial Formulation
    \item De Casteljau Recursive Interpolation Algorithm
De Casteljau’s algorithm is numerically stable and geometrically intuitive.
A Bézier curve of degree $n$ is defined using Bernstein basis polynomials:
\[
B_{k,n}(t) = \binom{n}{k} t^k (1 - t)^{n-k}
\]
The general Bézier curve equation is:
\[
P(t) = \sum_{k=0}^{n} B_{k,n}(t) P_k
\]
% ==========================================================
% CHAPTER 3: METHODOLOGY
% ==========================================================
\chapter{Methodology}

\section{System Overview}

The project consists of four main modules:

\begin{itemize}
    \item \textbf{UI Module (ui.py)} – Tkinter-based launcher that scans and executes project files.
    \item \textbf{Complex Transformation Module (pre.py)} – Visualizes rotation and scaling in the complex plane.
    \item \textbf{2D Bézier Module (bezier.py)} – Implements De Casteljau algorithm with draggable control points.
    \item \textbf{3D Bézier Solver (bezierSolve.py)} – Computes Bernstein polynomial form and generates parametric equations.
\end{itemize}

\section{Mathematical Foundation}

\subsection{De Casteljau Algorithm}

Recursive interpolation formula:

\[
P_i^k = (1 - t)P_i^{k-1} + tP_{i+1}^{k-1}
\]

Repeated until a single point remains.

\subsection{Bernstein Polynomial Basis}

\[
B_{k,n}(t) = \binom{n}{k} t^k (1 - t)^{n-k}
\]

\[
P(t) = \sum_{k=0}^{n} B_{k,n}(t) P_k
\]

\subsection{Complex Rotation}

If $z = x + iy$, then multiplying by $i$ rotates the vector by 90 degrees counterclockwise.

\section{Algorithms Used}

\subsection{De Casteljau Algorithm}
Implemented in \texttt{bezier.py} for 2D curve construction.

\subsection{Bernstein Polynomial Evaluation}
Implemented in \texttt{bezierSolve.py} for 3D parametric generation.

\subsection{Interactive Event Handling}
Mouse drag events update control points dynamically.

\subsection{Slider-Based Parameter Control}
Parameter $t$ varies from 0 to 1 to trace the curve.

\section{Implementation Snippet}

\begin{minted}[fontsize=\small, linenos]{python}
def bezier_parametric_eq(points):
    n = len(points) - 1
    x_eq = " + ".join([f"{comb(n,k)}*{points[k,0]}*t**{k}*(1-t)**{n-k}" for k in range(n+1)])
    y_eq = " + ".join([f"{comb(n,k)}*{points[k,1]}*t**{k}*(1-t)**{n-k}" for k in range(n+1)])
    z_eq = " + ".join([f"{comb(n,k)}*{points[k,2]}*t**{k}*(1-t)**{n-k}" for k in range(n+1)])
    return x_eq, y_eq, z_eq


x_eq, y_eq, z_eq = bezier_parametric_eq(control_points)

print("\nParametric equations (Bernstein form):")
print(f"x(t) = {x_eq}")
print(f"y(t) = {y_eq}")
print(f"z(t) = {z_eq}")
\end{minted}
\section{Sample Table}

\begin{table}[htb]
\centering
\begin{tabular}{|c|c|c|}
\hline
\textbf{Control Points} & \textbf{Degree} & \textbf{Curve Type} \\
\hline
2 & 1 & Linear \\
3 & 2 & Quadratic \\
4 & 3 & Cubic \\
5 & 4 & Quartic \\
\hline
\end{tabular}
\caption{Relationship between control points and curve degree}
\end{table}
\newpage
\section{Screenshots / Figures}
\begin{figure}[htbp]
    \centering
    \includegraphics[width=0.8\textwidth]{images/ui_launcher.png}
    \caption{Main UI Launcher (ui.py)}
\end{figure}

\begin{figure}[htbp]
    \centering
    \includegraphics[width=0.8\textwidth]{images/bezier2d.png}
    \caption{2D Bézier Curve with Control Points}
\end{figure}

\begin{figure}[htbp]
    \centering
    \includegraphics[width=0.8\textwidth]{images/bezier3d.png}
    \caption{3D Bézier Curve Visualization}
\end{figure}

\begin{figure}[htbp]
    \centering
    \includegraphics[width=0.8\textwidth]{images/pre.png}
    \caption{Interactive Prerequisite with Mcq}
\end{figure}
% ==========================================================
% CHAPTER 4: RESULT AND ANALYSIS
% ==========================================================
\chapter{Result and Analysis}

\section{Correctness of Rendering Output}
The Bézier curves pass through the first and last control points as expected. The curve remains inside the convex hull of the control polygon, validating correctness.
The 3D visualization correctly follows the parametric equations generated.

\section{Transformation Behavior}
In \texttt{pre.py}:
\begin{itemize}
    \item Stretch slider changes magnitude only.
    \item Turn slider changes angle only.
    \item Four successive multiplications by $i$ return to original orientation.
\end{itemize}
In \texttt{bezier.py}:
\begin{itemize}
    \item Increasing control points increases curve flexibility.
    \item Parameter $t$ smoothly traces the curve.
\end{itemize}
\section{Performance Discussion}
The system runs smoothly for up to 8 control points in real time. Slider updates and drag events are responsive due to efficient NumPy computation.
\section{Limitations}
\begin{itemize}
    \item Derivations aren't focused as it's intuition central.
    \item No export of curve data.
    \item 3D input is terminal-based instead of GUI.
    \item MCQs lacking in exam oriented mastery.
\end{itemize}

% ==========================================================
% CHAPTER 5: CONCLUSION AND FUTURE ENHANCEMENT
% ==========================================================
\chapter{Conclusion and Future Enhancement}
\section{Conclusion}
This project successfully integrates mathematical theory and visualization. It demonstrates how complex transformations and Bézier curves can be implemented interactively using Python.
The modular structure allows easy extension and experimentation.
\section{Future Enhancements}
\begin{itemize}
    \item Implement Bézier surface patches.
    \item Add GUI-based 3D control point input.
    \item Export curve as image or animation.
    \item Make the questions and interaction more geared towards course oriented mastery
\end{itemize}

% ==========================================================
% APPENDIX
% ==========================================================
\appendix
\chapter{Appendices}
\section{UI Controls}
\begin{itemize}
    \item Mouse Click: Add control point
    \item Key R: Reset
    \item Key Q: Exit
\end{itemize}
% ==========================================================
% REFERENCES
% ==========================================================
\bibliographystyle{ieeetr}
\bibliography{includes/ref}
% use \cite{} to cite the part taken as reference
\cite{mathworld_complex}
\cite{farin_bezier}
\cite{class_notes}
\end{document}